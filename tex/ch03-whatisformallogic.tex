\chapter{What is Formal Logic?}
\label{chap:whatisformallogic}
\markright{Ch. \ref{chap:whatisformallogic}: What is Formal Logic?}
\setlength{\parindent}{1em}

% **************************************************
% *				12.1 Constructing Formal Languages             *
% **************************************************

\section{Constructing Formal Languages}


The chapters in 
\iflabelexists{part:formal_logic}{Part \ref{part:formal_logic}} %This prints ``Part $N$ if there is a single section for all of formal logic
{Parts \iflabelexists{part:cat_logic}{\ref{part:cat_logic} and \ref{part:sent_logic}} %this prints ``Parts $N$ and $M'$' if there is a section for cat logic, where $N$ and $M$ are the part numbers for cat and sent logic.
{\ref{part:sent_logic} and \ref{part:quant_logic}}} %this prints ``Parts $N$ and $M'$' if there is not a section for cat logic, where $N$ and $M$ are the part numbers for sent and quant logic.
deal with formal logic. Formal logic is distinguished from other branches of logic by the way it achieves content neutrality. Back on page \pageref{def:content_neutrality}, we said that a distinctive feature of logic is that it is neutral about the content of the argument it evaluates. If a kind of argument is strong---say, a kind of statistical argument---it will be strong whether it is applied to sports, politics, science or whatever. Formal logic takes radical measures to ensure content neutrality: it removes the parts of a statement that tie it to particular objects in the world and replaces them with variables.

Consider the two arguments from Figure \ref{fig:valid_sound} again:
\begin{multicols}{2}
\begin{earg*}
\item Socrates is a person.
\item All persons are mortal.
\itemc Socrates is mortal.
\end{earg*}

\begin{earg*}
\item Socrates is a person.
\item All people are carrots.
\itemc Socrates is a carrot.
\end{earg*}

\end{multicols}

These arguments are both valid. In each case, if the premises were true, the conclusion would have to be true. (In the case of the first argument, the premises are actually true, so the argument is sound, but that is not what we are concerned with right now.) These arguments are valid, because they are put together the right way. Another way of thinking about this is to say that they have the same logical form. Both arguments can be written like this:

\begin{earg*}
\item $S$ is $M$.
\item All $M$ are $P$.
\itemc[.2] $S$ is $P$.
\end{earg*}

In both arguments $S$ stands for Socrates and $M$ stands for person. In the first argument, $P$ stands for mortal; in the second, $P$ stands for carrot. \iflabelexists{chap:catstatements}{(The reason we chose these letters will become clear in Chapters \ref{chap:catstatements} and \ref{chap:cat_syllogisms}.)}{} The letters `S', `M', and `P' are variables. They are just like the variables you may have learned about in algebra class. In algebra, you had equations like $y = 2x + 3$, where $x$ and $y$ were variables that could stand for any number. Just as $x$ could stand for any number in algebra, `S' can stand for any name in logic. In fact, this is the original use of variables. Long before variables were used to stand for numbers in algebra, they were used to stand for classes by Aristotle in his book the \textit{Prior Analytics} (c. 350 \textsc{bce}/1\citeyear{Aristotle1984b}). There are no recorded instances of people using variables before this, but Aristotle uses them fairly casually, as if his readers would be familiar with the idea, so it may be that someone prior to Aristotle actually invented the variable.

The invention of the variable was one of the most important conceptual innovations in human history, right up there with the invention of the zero, or alphabetic writing. The importance of the variable for the history of mathematics is obvious. But it was also incredibly important in its original field of application, logic. For one thing, it allows logicians to be more content neutral. We can set aside any associations we have with people, or carrots, or whatever, when we are analyzing an argument. More importantly, once we set aside content in this way, we discover that something incredibly powerful is left over, the logical structure of the sentence itself. This is what we investigate when we study formal logic.In the case of the two arguments above, identifying the logical structure of statements reveals not only that the two arguments have the same logical form, but they have an impeccable logical form. Both arguments are valid, and any other arguments that have this form will be valid. 

When Aristotle introduced the variable he used it the way we did in the argument above. His variables stood for names and categories in simple two-premise arguments called syllogisms. The system of logic Aristotle outlined became the dominant logic in the Western world for more than two millennia. It was studied and elaborated on by philosophers and logicians from Baghdad to Paris. The thinkers that carried on Aristotelian tradition were divided by language and religion. They were pagans, Muslims, Jews, and Christians writing typically in Greek, Latin or Arabic. But they were all united by the sense that the tools Aristotle had given them allowed them to see something profound about the nature of reality.

\newglossaryentry{artificial language}
{
name=artificial language,
description={A language that was consciously developed by identifiable individuals for some purpose.}
}

\newglossaryentry{natural language}
{
name=natural language,
description={A language that develops spontaneously and learned by infants as their first language.}
}

\newglossaryentry{formal language}
{
name=formal language,
description={An artificial language designed to bring out the logical structure of  ideas and remove all the ambiguity and vagueness that plague natural languages like English.}
}

Despite its historical importance, Aristotelean logic has largely been superseded. Starting in the 19th century people learned to do more than simply replace categories with variables. They learned to replicate the whole structure of sentences with a formal system that brought out all sorts of features of the logical form of arguments. The result was the creation of entire artificial languages. An \textsc{\gls{artificial language}} \label{def:artificial_language} is a language that was consciously developed by identifiable individuals for some purpose. Esperanto, for instance, is an artificial language developed by Ludwig Lazarus Zamenhof in the 19th century with the hope of promoting world peace by creating a common language for all. Artificial languages contrast with \textsc{\glspl{natural language}}, \label{def:natural_language} which develop spontaneously and are learned by infants as their first language. 

The languages developed by logicians are specifically formal languages. A \textsc{\gls{formal language}} \label{def:formal_language} is an artificial language designed to bring out the logical structure of ideas and remove all the ambiguity and vagueness that plague natural languages like English. Creating formal languages always involves a trade off. % replace ``a trade off'' with something that will introduce both the simplicity/scope trade off and the necessity of ambituity.

%Say something about the necessity of ambiguity here. 

On the one hand, we want to simplify the language so we can focus on just the parts of arguments that make them valid or invalid. If we included every feature of the English language, all of the subtlety and nuance, then there would be no advantage in translating to a formal language. We might as well think about the argument in English. At the same time, we would like a formal language that allows us to represent many kinds of natural language arguments. One reason Aristotelian logic was superseded was that it could only capture a tiny subset of the possible valid arguments. 

So when deciding on a formal language, there is inevitably a tension between wanting to capture as much structure as possible and wanting a simple formal language---simpler formal languages leave out more. This means that there is no perfect formal language. Some will do a better job than others in translating particular English-language arguments.

Formal logic has traditionally been studied for theoretical reasons: people were trying to understand the nature of reason itself. However, formal logic has given birth to at least one world-changing practical application, the digital computer. Modern computer science began with the work of the logician Alan Turing. In the 1930s, he developed the idea of a reasoning machine that could compute any function. At first, this was just an abstract idea, but during World War II, Turing worked for the British code breaking effort at Bletchley Park. There his imaginary logic machines became real, and were crucial for breaking the Nazi Enigma code. All modern computers are descended from Alan Turing's work in logic, and all modern computer languages are descended from the formal languages developed by logicians. Thus even at its most abstract, formal logic has had profound practical applications.

In the chapters to come, you will learn more precise ways to replace English words with abstract letters and symbols to see their formal structure. In many ways, these letters and symbols will function a lot like mathematical symbols. Also, to a large extent, our paradigm of a good argument will be a mathematical proof. As a result, much of what we will study in this book will have the same puzzle solving character that one finds in mathematics. 

%The next paragraph gives an overview of the chapters to come. It comes in three forms, based on common configurations of the textbook

%This version is for the complete text, where all formal sections are covered in a unified Part II.
\iflabelexists{part:formal_logic}{Part \ref{part:formal_logic} of this book begins by exploring the world of Aristotelian logic. Chapter \ref{chap:catstatements} looks at the logical structure of the individual statements studied by the Aristotelian tradition. Chapter \ref{chap:cat_syllogisms} then builds these into valid arguments. After we study Aristotelian logic, we will develop two formal languages, called SL and QL.  Chapters \ref{chap:SL} through \ref{chap:proofsinSL} develop SL. In SL, the smallest units are individual statements. Simple statements are represented as letters and connected with {logical connectives} like \emph{and} and \emph{not} to make more complex statements. Chapters \ref{chap:QL} through \ref{chap:proofsinQL} develop QL. In QL, the basic units are objects, properties of objects, and relations between objects.}{
%% This version of the paragraph is for texts that just do cat and sent.
\iflabelexists{part:cat_logic}{Part \ref{part:cat_logic} of this book explores the world of Aristotelian logic. Chapter \ref{chap:catstatements} looks at the logical structure of the individual statements studied by the Aristotelian tradition. Chapter \ref{chap:cat_syllogisms} then builds these into valid arguments. Part \ref{part:sent_logic} develops a full-blown formal language, called Sentential logic, or SL. In SL Simple statements are represented as letters and connected with logical connectives like \emph{and} and \emph{not} to make more complex statements.}{
%This version of the paragraph is for texts that just do sent and quant
In this book we will be developing two formal languages, called SL and QL. Part \ref{part:sent_logic} develops SL.In SL, the smallest units are individual statements. Simple statements are represented as letters and connected with logical connectives like \emph{and} and \emph{not} to make more complex statements. Part \ref{part:quant_logic} develops QL. In QL, the basic units are objects, properties of objects, and relations between objects.} 




}




%\iflabelexists{part:formal_logic}{ and QL.  Chapters \ref{chap:SL} through \ref{chap:proofsinSL} develop SL. In SL, the smallest units are individual statements. Several chapters in the complete version of this text \label{ver_var}\nix{Chapters \ref{chap:QL} through \ref{chap:proofsinQL}} develop QL. In QL, the basic units are objects, properties of objects, and relations between objects.


% **********************************************
% *			More Logical Notions for Formal Logic      *
% **********************************************
\section{More Logical Notions for Formal Logic}
\label{sec:other_logical_notions}
\setlength{\parindent}{1em}

Part \ref{part:basic_concepts} covered the basic concepts you need to study any kind of logic. When we study formal logic, we will be interested in some additional logical concepts, which we will explain here. 

%1.5.1 Truth values

\subsection{Truth values}

\newglossaryentry{truth value}
{
  name=truth value,
  description={The status of a statement with relationship to truth. For  this textbook, this means the status of a statement as true or false}
}


A truth value is the status of a statement as true or false. Thus the truth value of the sentence ``All dogs are mammals'' is ``True,'' while the truth value of ``All dogs are reptiles'' is false. More precisely, a \textsc{\gls{truth value}} \label{def:Truth_value} is the status of a statement with relationship to truth. We have to say this, because there are systems of logic that allow for truth values besides ``true'' and ``false,'' like ``maybe true,'' or ``approximately true,'' or ``kinda sorta true.'' For instance, some philosophers have claimed that the future is not yet determined. If they are right, then statements about \emph{what will be the case} are not yet true or false. Some systems of logic accommodate this by having an additional truth value. Other formal languages, so-called paraconsistent logics, allow for statements that are both true \emph{and} false. We won't be dealing with those in this textbook, however. For our purposes, there are two truth values, ``true'' and ``false,'' and every statement has exactly one of these. Logical systems like ours are called \define{bivalent}. \label{defBivalent}







%1.5.2 Tautology, Contingent Statement, Contradiction

\subsection{Tautology, contingent statement, contradiction}

In considering arguments formally, we care about what would be true \emph{if} the premises were true. Generally, we are not concerned with the actual truth value of any particular statements--- whether they are \emph{actually} true or false. Yet there are some statements that must be true, just as a matter of logic.

Consider these statements:
\begin{enumerate}[label=(\alph*)]
\item \label{itm:ex_contingent} It is raining.
\item \label{itm:ex_tautology} Either it is raining, or it is not.
\item \label{itm:ex_contradiction} It is both raining and not raining.
\end{enumerate}
In order to know if statement \ref{itm:ex_contingent} is true, you would need to look outside or check the weather channel. Logically speaking, it might be either true or false. Statements like this are called \emph{contingent} statements.


\newglossaryentry{tautology}
{
name=tautology,
description={A statement that must be true, as a matter of logic.}
}

Statement \ref{itm:ex_tautology} is different. You do not need to look outside to know that it is true. Regardless of what the weather is like, it is either raining or not. If it is drizzling, you might describe it as partly raining or in a way raining and a way not raining. However, our assumption of bivalence means that we have to draw a line, and say at some point that it is raining. And if we have not crossed this line, it is not raining. Thus the statement ``either it is raining or it is not'' is always going to be true, no matter what is going on outside. A statement that has to be true, as a matter of logic is called a \textsc{\gls{tautology}} \label{def:tautology} or logical truth. 

\newglossaryentry{contradiction}
{
name=contradiction,
description={A statement that must be false, as a matter of logic.}
}

You do not need to check the weather to know about statement \ref{itm:ex_contradiction}, either. It must be false, simply as a matter of logic. It might be raining here and not raining across town, it might be raining now but stop raining even as you read this, but it is impossible for it to be both raining and not raining here at this moment. The third statement is \emph{logically false}; it is false regardless of what the world is like. A logically false statement is called a \textsc{\gls{contradiction}}. \label{def:contradiction}

\newglossaryentry{contingent statement}
{
name=contingent statement,
description={A statement that is neither a tautology nor a contradiction.}
}

We have already said that a contingent statement is one that could be true, or could be false, as far as logic is concerned. To be more precise, we should define a \textsc{\gls{contingent statement}}  \label{def:contingent_statement} as a statement that is neither a tautology nor a contradiction. This allows us to avoid worrying about what it means for something to be logically possible. We can just piggyback on the idea of being logically necessary or logically impossible. 

A statement might \emph{always} be true and still be contingent. For instance, it may be the case that in no time in the history of the universe was there ever an elephant with tiger stripes. Elephants only ever evolved on Earth, and there was never any reason for them to evolve tiger stripes. The statement ``Some elephants have tiger stripes,'' is therefore always false. It is, however, still a contingent statement. The fact that it is always false is not a matter of logic. There is no contradiction in considering a possible world in which elephants evolved tiger stripes, perhaps to hide in really tall grass. The important question is whether the statement \emph{must} be true, just on account of logic.

When you combine the idea of tautologies and contradictions with the notion of deductive validity, as we have defined it, you get some curious results. For one thing, any argument with a tautology in the conclusion will be valid, even if the premises are not relevant to the conclusion. This argument, for instance, is valid.

\begin{earg*}
\item There is coffee in the coffee pot.
\item There is a dragon playing bassoon on the armoire.
\itemc All bachelors are unmarried men.
\end{earg*}

The statement ``All bachelors are unmarried men'' is a tautology. No matter what happens in the world, all bachelors have to be unmarried men, because that is how the word ``bachelor'' is defined. But if the conclusion of the argument is a tautology, then there is no way that the premises could be true and the conclusion false. So the argument must be valid.

Even though it is valid, something seems really wrong with the argument above. The premises are not relevant to the conclusion. Each sentence is about something completely different. This notion of relevance, however, is something that we don't have the ability to capture in the kind of simple logical systems we will be studying. The logical notion of validity we are using here will not capture everything we like about arguments.

Another curious result of our definition of validity is that any argument with a contradiction in the premises will also be valid. In our kind of logic, once you assert a contradiction, you can say anything you want. This is weird, because you wouldn't ordinarily say someone who starts out with contradictory premises is arguing well. Nevertheless, an argument with contradictory premises is valid.

%1.5.3 Logical equivalence. 

\subsection{Logically Equivalent and Contradictory Pairs of Sentences}

We can also ask about the logical relations \emph{between} two statements. For example:

\begin{enumerate}[label=(\alph*)]
\item John went to the store after he washed the dishes.
\item John washed the dishes before he went to the store.
\end{enumerate}

\newglossaryentry{logical equivalence}
{
name={logical equivalence},
text={logically equivalent},
description={A property held by a pair of sentences that must always have the same truth value.}
}

These two statements are both contingent, since John might not have gone to the store or washed dishes at all. Yet they must have the same truth value. If either of the statements is true, then they both are; if either of the statements is false, then they both are. When two statements necessarily have the same truth value, we say that they are \textsc{\gls{logical equivalence}}. \label{def:logical_equivalence}

\newglossaryentry{contradictories}
{
name=contradictories,
description={Two statements that must have opposite truth values, so that one must true and the other false.}
}

On the other hand, if two sentences must have opposite truth values, we say that they are \textsc{\gls{contradictories}}. \label{def:contradictory}Consider these two sentences 

\begin{enumerate}[label=(\alph*)]
\item Susan is taller than Monica.
\item Susan is shorter or the same height as Monica.
\end{enumerate}

One of these sentences must be true, and if one of the sentences is true, the other one is false. It is important to remember the difference between a single sentence that is a \emph{contradiction} and a pair of sentences that are \emph{contradictory}. A single sentence that is a contradiction is in conflict with itself, so it is never true. When a pair of sentences is contradictory, one must always be true and the other false.

%%%%%%%%%%%%%%  consistency

\subsection{Consistency}
Consider these two statements:

\begin{enumerate}[label=(\alph*)]
\item \label{itm:taller} My only brother is taller than I am.
\item \label{itm:shorter} My only brother is shorter than I am.
\end{enumerate}

Logic alone cannot tell us which, if either, of these statements is true. Yet we can say that \emph{if} the first statement \ref{itm:taller} is true, \emph{then} the second statement \ref{itm:shorter} must be false. And if \ref{itm:shorter}  is true, then \ref{itm:taller} must be false. It cannot be the case that both of these statements are true. It is possible, however that both statements can be false. My only brother could be the same height as I am. 

\newglossaryentry{inconsistency}
{
name=inconsistency,
text={inconsistent},
description={A property possessed by a set of sentences when they cannot all be true at the same time, but they may all be false at the same time.}
}

\newglossaryentry{consistency}
{
name=consistency,
text={consistent},
description={A property possessed by a set of sentences when they can all be true at the same time, but are not necessarily so.}
}

If a set of statements could not all be true at the same time, they are said to be \textsc{\gls{inconsistency}}. \label{def:inconsistency} Otherwise, they are \textsc{\gls{consistency}}. \label{def:consistency} 

We can ask about the consistency of any number of statements. For example, consider the following list of statements:

\label{MartianGiraffes}
\begin{enumerate}[label=(\alph*)]
\item \label{itm:at_least_four}There are at least four giraffes at the wild animal park.
\item \label{itm:exactly_seven} There are exactly seven gorillas at the wild animal park.
\item \label{itm:not_more_than_two} There are not more than two Martians at the wild animal park.
\item \label{itm:martians} Every giraffe at the wild animal park is a Martian.
\end{enumerate}

Statements \ref{itm:at_least_four} and \ref{itm:martians} together imply that there are at least four Martian giraffes at the park. This conflicts with \ref{itm:not_more_than_two}, which implies that there are no more than two Martian giraffes there. So the set of statements \ref{itm:at_least_four}--\ref{itm:martians} is inconsistent. Notice that the inconsistency has nothing at all to do with \ref{itm:exactly_seven}. Statement \ref{itm:exactly_seven} just happens to be part of an inconsistent set.

Sometimes, people will say that an inconsistent set of statements ``contains a contradiction.'' By this, they mean that it would be logically impossible for all of the statements to be true at once. A set can be inconsistent even when all of the statements in it are either contingent or tautologous. When a single statement is a contradiction, then that statement alone cannot be true.

%%%%%%%%%%%  Practice Problems %%%%%%%%%%%


\practiceproblems
\noindent \problempart \label{pr.EnglishTautology} Label the following tautology, contradiction, or contingent statement.

\begin{longtabu}{p{.1\linewidth}p{.9\linewidth}}
\textbf{Example}: & Caesar crossed the Rubicon. \\
\textbf{Answer}: & Contingent statement. \\
&(The Rubicon is a river in Italy. When General Julius Caesar took his army across it, he was committing to a revolution against the Roman Republic. Since that time, ``crossing the Rubicon'' has been a expression referring to making an irreversible decision.)\\
\end{longtabu}

\begin{exercises}
\item Someone once crossed the Rubicon. \answer{\underline{Contingent statement}}
\item No one has ever crossed the Rubicon. \answer{\underline{Contingent  statement}}
\item If Caesar crossed the Rubicon, then someone has. \answer{\underline{Tautology}}
\item Even though Caesar crossed the Rubicon, no one has ever crossed the Rubicon. \answer{\underline{Contradiction}}
\item If anyone has ever crossed the Rubicon, it was Caesar. \answer{\underline{Contingent statement}}
\end{exercises}

\noindent \problempart Label the following tautology, contradiction, or contingent statement.
\begin{exercises}
\item Elephants dissolve in water. \answer{\underline{Contingent}}
\item Wood is a light, durable substance useful for building things. \answer{\underline{Contingent}}
\item If wood were a good building material, it would be useful for building things. \answer{\underline{Tautology}}
\item I live in a three story building that is two stories tall. \answer{\underline{Contradiction}}
\item If gerbils were mammals they would nurse their young. \answer{\underline{Tautology}}
\end{exercises}

\noindent \problempart Which of the following pairs of statement are logically equivalent? 

\begin{exercises}
\item Elephants dissolve in water.	\\
	If you put an elephant in water, it will disintegrate.
\answer{\\\underline{Logically equivalent}}	
\item All mammals dissolve in water.\\		
	If you put an elephant in water, it will disintegrate. 
\answer{\\ \underline{Not logically equivalent}}
\item George Bush was the 43rd president. \\
	 Barack Obama is the 44th president. 
\answer{\\\underline{Not logically equivalent}}
\item Barack Obama is the 44th president. \\
	  Barack Obama was president immediately after the 43rd president. 
\answer{\\ \underline{Logically equivalent}}
\item Elephants dissolve in water. 	\\	
	All mammals dissolve in water. 
\answer{\\ \underline{Not logically equivalent}}
\end{exercises}


% replace this with a ``logically equivalent, contradictory or neither'' exercise.

\noindent \problempart Which of the following pairs of statement are logically equivalent? 

\begin{exercises}
\item  Thelonious Monk played piano.	\\
	John Coltrane played tenor sax. 
	\answer{\\ \underline{Not logically equivalent}}
\item  Thelonious Monk played gigs with John Coltrane.	\\
	John Coltrane played gigs with Thelonious Monk.
\answer{\\ \underline{Logically equivalent}}
\item  All professional piano players have big hands.	\\
	Piano player Bud Powell had big hands.
	\answer{\\ \underline{Not logically equivalent}}
\item  Bud Powell suffered from severe mental illness.	 \\
	All piano players suffer from severe mental illness.
	\answer{\\ \underline{Not logically equivalent}}
\item John Coltrane was deeply religious.	 \\
John Coltrane viewed music as an expression of spirituality. 
\answer{\\ \underline{Not logically equivalent}}
\end{exercises}

% replace this with a ``logically equivalent, contradictory or neither'' exercise.

\noindent \problempart Consider again the statements on p.\pageref{MartianGiraffes}: 
\begin{enumerate}[label=(\alph*)]
\item \label{itm:at_least_four}There are at least four giraffes at the wild animal park.
\item \label{itm:exactly_seven} There are exactly seven gorillas at the wild animal park.
\item \label{itm:not_more_than_two} There are not more than two Martians at the wild animal park.
\item \label{itm:martians} Every giraffe at the wild animal park is a Martian.
\end{enumerate}
Now consider each of the following sets of statements. Which are consistent? Which are inconsistent?
\begin{exercises}
\item Statements \ref{itm:exactly_seven}, \ref{itm:not_more_than_two}, and \ref{itm:martians} \answer{\underline{consistent}}
\item Statements \ref{itm:at_least_four}, \ref{itm:not_more_than_two}, and \ref{itm:martians} \answer{\underline{inconsistent}}
\item Statements \ref{itm:at_least_four}, \ref{itm:exactly_seven}, and \ref{itm:martians}\answer{\underline{consistent}}
\item Statements \ref{itm:at_least_four}, \ref{itm:exactly_seven}, and \ref{itm:not_more_than_two} \answer{\underline{consistent}}
\end{exercises}

\noindent \problempart Consider the following set of statements.
\begin{enumerate}[label=(\alph*)]
\item \label{itm:allmortal} All people are mortal.
\item \label{itm:socperson} Socrates is a person.
\item \label{itm:socnotdie} Socrates will never die.
\item \label{itm:socmortal} Socrates is mortal.
\end{enumerate}
Which combinations of statements form consistent sets? Mark each “consistent” or “inconsistent.”
\begin{exercises}
\item Statements \ref{itm:allmortal}, \ref{itm:socperson}, and \ref{itm:socnotdie}  \answer{\underline{Inconsistent}}
\item Statements \ref{itm:socperson}, \ref{itm:socnotdie}, and \ref{itm:socmortal} \answer{\underline{Inconsistent}}
\item Statements \ref{itm:socperson} and \ref{itm:socnotdie} \answer{\underline{Consistent}}
\item Statements \ref{itm:allmortal} and \ref{itm:socmortal} \answer{\underline{Consistent}}
\item Statements \ref{itm:allmortal}, \ref{itm:socperson}, \ref{itm:socnotdie}, and \ref{itm:socmortal} \answer{\underline{Inconsistent}} 
\end{exercises}

\noindent \problempart \label{pr.EnglishCombinations} Which of the following is possible? If it is possible, give an example. If it is not possible, explain why.
\begin{exercises}
\item A valid argument that has one false premise and one true premise

\answer{\underline{Possible}. Example: If grass is green, then I am the pope. (False) Grass is green. (True) \therefore  I am the pope. (False) \\

Remember, if an argument is valid, the only thing that can't happen is for it to have all true premises and a false conclusion. So if you don't specify a false conclusion anything is possible. \\}

\item A valid argument that has a false conclusion

\answer{\underline{Possible}. Same example as above. \\}

\item A valid argument, the conclusion of which is a contradiction

\answer{\underline{Possible}. The conclusion is always false, but if the premises are also always false, you are fine. Example: If A, then not A. \therefore If B, then not B. \\}

\item An invalid argument, the conclusion of which is a tautology

\answer{\underline{Impossible}. If the conclusion is always true, then the there is no way for all the premises to be true and conclusion false.\\}

\item A tautology that is contingent

\answer{\underline{Impossible}. Contradictions, contingencies, and tautologies are exclusive categories. If you are one, you can't be either of the others. \\}


\item Two logically equivalent sentences, both of which are tautologies

\answer{\underline{Possible} In fact, all tautologies are logically equivalent. Logically equivalent sentences always have the same truth value, and all tautologies are always true. \\}


\item Two logically equivalent sentences, one of which is a tautology and one of which is contingent

\answer{\underline{Impossible}. A tautology is always true, but contingent sentences can be false. Therefore they can have different truth values. \\}


\item Two logically equivalent sentences that together are an inconsistent set

\answer{\underline{Possible} Two contradictions are logically equivalent, however it is impossible for them to both be true, because it is impossible for either one to be true. \\}


\item A consistent set of sentences that contains a contradiction

\answer{\underline{Impossible}. The contradiction can never be true, so the whole set cannot never all be true. \\}


\item An inconsistent set of sentences that contains a tautology
\answer{\underline{Possible}. Example: A, Not A, If A then A.} 
\end{exercises}

\noindent \problempart Which of the following is possible? If it is possible, give an example. If it is not possible, explain why.
\answer{Answers by Ben Sheredos}
\begin{exercises}
\item A valid argument, whose premises are all tautologies, and whose conclusion is contingent
\answer{Not Possible. If the argument is valid, then the conclusion must be true if the premises are true. If the premises are \textit{tautologies}, then the premises are \textit{always} true, and so the conclusion also must always be true.}

\item A valid argument with true premises and a false conclusion
\answer{ \textit{Absolutely not!} This contradicts the very definition of a valid argument.
}
\item A consistent set of sentences that contains two sentences that are not logically equivalent
\answer{ Most definitely. Here are two sentences that are consistent but not logically equivalent: ``Today is a Wednesday'' and ``I like pie.''
}
\item A consistent set of sentences, all of which are contingent
\answer{For sure. See the examples given in the previous answer. Both are contingent (sometimes it's not Wednesday today, and I might've hated pie.)
}
\item A false tautology
\answer{Not possible. By definition, a tautology is always true.
}
\item A valid argument with false premises
\answer{ Yup. Because validity only requires that \textit{if} the premises are true, \textit{then} the conclusion must be true. But all of them could be false, and the argument would remain valid. 
}
\item A logically equivalent pair of sentences that are not consistent
\answer{ Careful here. Our definition of consistency is that a set of statements are consistent if they could all be true at the same time. Well, consider the case of 2 statements which are logically equivalent, and which are both \textit{contradictions}. Neither can be true. So they cannot \textit{both} be true. So they are not consistent. 
}
\item A tautological contradiction
\answer{ Impossible. This is gibberish-nonsense.
}
\item A consistent set of sentences that are all contradictions
\answer{ Nope: see again \#7 above. If a set of statements contains nothing but contradictions, then none of them can be true. But if none of them can be true, then they cannot be true together, and so they cannot be consistent.
}
\end{exercises}

\section*{Key Terms}
\begin{sortedlist}
\sortitem{Truth value}{} 	
\sortitem{Natural language}{}
\sortitem{Artificial language}{}
\sortitem{Formal language}{}
\sortitem{Tautology}{}
\sortitem{Contradiction}{}
\sortitem{Contingent statement}{}
\sortitem{Logically equivalent}{}
\sortitem{Contradictories}{}
\sortitem{Consistent}{}
\sortitem{Inconsistent}{}

\end{sortedlist}
